% ---------------------------------------------------------------------------- %
\subsection{Versuch 3.3 -- Planck-Konstante}
\label{subsec:planck}
% ---------------------------------------------------------------------------- %

Die manuelle Auswertung des Scatter-Plots f\"ur den Grenzwinkel $\vartheta_G$ ist in
Tabelle~\ref{tab:resultsPlanckManualFit} zusammengefasst.

\begin{table}[h!]
    \centering
    \small
    \caption{Grenzwinkel $\vartheta_G$}~\label{tab:resultsPlanckManualFit}
    \begin{tabular}{rrr}
        \toprule
        Spannung U          & Winkel $2 \cdot \vartheta_G$    & Winkel $\vartheta_G$             \\
        \midrule
        \SI{35}{\kilo\volt} &       \SI{5     \pm~1   }{\degree} & \SI{2.5   \pm~0.5  }{\degree} \\
        \SI{32}{\kilo\volt} &       \SI{7.5   \pm~0.5 }{\degree} & \SI{3.75  \pm~0.25 }{\degree} \\
        \SI{26}{\kilo\volt} &       \SI{10    \pm~1   }{\degree} & \SI{5     \pm~0.5  }{\degree} \\
        \SI{17}{\kilo\volt} &       \SI{19.75 \pm~0.25}{\degree} & \SI{9.875 \pm~0.125}{\degree} \\
        \SI{14}{\kilo\volt} &       \SI{23.5  \pm~0.5 }{\degree} & \SI{11.75 \pm~0.25 }{\degree} \\
        \SI{12}{\kilo\volt} &       \SI{28.5  \pm~0.5 }{\degree} & \SI{14.25 \pm~0.25 }{\degree} \\
        \bottomrule
    \end{tabular}
\end{table}

Mithilfe der Bragg'schen Reflexionsgleichung, aufgel\"ost nach der Wellenl\"ange $\lambda_G$:

\begin{equation}
    \label{eq:BraggSolvedPlanck}
    \lambda_G = \frac{2 \cdot d \cdot sin(\vartheta_G)}{n}
\end{equation}

und $n = 1$ sowie $d  = \SI{201.5}{\pico\meter}$ kann man nun die zuge\"origen
Wellenl\"angen bestimmen. Zur Bestimmung der Unsicherheiten werden diese dabei
einfach ebenfalls in Gleichung \ref{eq:BraggSolvedPlanck} eingesetzt. Die Ergebnisse
sind in Tabelle \ref{tab:resultsPlanckWavelengths} zu sehen.

\begin{table}[h!]
    \centering
    \small
    \caption{Wellenl\"angen mit Unsicherheiten f\"ur die Grenzwinkel aus Tabelle~\ref{tab:resultsPlanckManualFit}}~\label{tab:resultsPlanckWavelengths}
    \begin{tabular}{rr}
        \toprule
        Spannung U          & Grenzwellenl\"ange $\lambda_G$     \\
        \midrule
        \SI{35}{\kilo\volt} &       \SI{17.58 \pm 3.52}{\pico\meter} \\
        \SI{32}{\kilo\volt} &       \SI{26.36 \pm 1.76}{\pico\meter} \\
        \SI{26}{\kilo\volt} &       \SI{35.12 \pm 3.52}{\pico\meter} \\
        \SI{17}{\kilo\volt} &       \SI{69.11 \pm 0.88}{\pico\meter} \\
        \SI{14}{\kilo\volt} &       \SI{82.07 \pm 1.76}{\pico\meter} \\
        \SI{12}{\kilo\volt} &       \SI{99.20 \pm 1.76}{\pico\meter} \\
        \bottomrule
    \end{tabular}
\end{table}


%
%\begin{figure}[ht!]
%\centering
%\begin{tikzpicture}
%    \begin{axis}[
%        width=.67\textwidth,
%        height=.3\textwidth,
%        %title = {Eisengehalt},
%        xlabel = {$I_0$ ($\si{\kilo\gram\meter\squared}$)},
%        %symbolic y coords = {ungewichtet,gewichtet, QtiPlot gewichtet},
%        symbolic y coords = {Referenz Versuchsanleitung,Versuch 3.1.2 Zylinder 2,Versuch 3.1.2 Zylinder,Versuch 3.1.2 Punktmasse,Versuch 3.1.1},
%    ]
%    \addplot+[
%        only marks,error bars/.cd,
%        x dir=both,x explicit,
%        error bar style={line width=0.5pt},
%        ]
%    coordinates {
%        (1.20e-2,Versuch 3.1.1) +- (1.9e-5,0)
%        (1.14e-2,Versuch 3.1.2 Punktmasse) +- (4e-4,0)
%        (1.13e-2,Versuch 3.1.2 Zylinder) +- (4e-4,0)
%        (1.23e-2,Versuch 3.1.2 Zylinder 2) +- (9e-5,0)
%        (1.16e-2,Referenz Versuchsanleitung) +- (0,0)
%    };
%    \end{axis}
%\end{tikzpicture}
%\caption{graphische Darstellung der Ergebnisse zum f\"ur $I_0$}
%\label{fig:resultsI0}
%\end{figure}
