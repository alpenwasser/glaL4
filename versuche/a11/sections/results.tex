%Es gibt  drei Parameter  zu evaluieren: Das Massentr\"agheitsmoment  $I_0$ der
%Apparatur ohne Reiter,  die Periodendauer $T_0$ f\"ur  kleine Auslenkungen und
%die Federkonstante $k$. Die numerischen Resultate  f\"ur $I_0$ sind in Tabelle
%\ref{tab:resultsI0}  zu  finden;  Abbildung  \ref{fig:resultsI0}  zeigt  einen
%graphischen Vergleich der verschiedenen Methoden.
%
%\begin{table}[h!]
%    \centering
%    \caption{Ergebnisse f\"ur $I_0$ via verschiedene Methoden}
%    \label{tab:resultsI0}
%    \begin{tabular}{p{70mm}r}
%        \toprule
%        Methode                                                                 & Resultat \\
%        \midrule
%        Versuch 3.1.1                                                           & \SI{12.03 \pm 0.02}{\gram\meter\squared} \\
%        Versuch 3.1.2, Reiter als Punktmasse approximiert                       & \SI{11.4 \pm 0.4}{\gram\meter\squared} \\
%        Versuch 3.1.2, Reiter als Zylinder modelliert                           & \SI{11.3 \pm 0.4}{\gram\meter\squared} \\
%        Versuch 3.1.2, Reiter als Zylinder modelliert, eingeengter Fit-Bereich  & \SI{12.31 \pm 0.09}{\gram\meter\squared} \\
%        Referenz Versuchsanleitung                                              & \SI{11.6}{\gram\meter\squared} \\
%        \bottomrule
%    \end{tabular}
%\end{table}
%
%\begin{figure}[ht!]
%\centering
%\begin{tikzpicture}
%    \begin{axis}[
%        width=.67\textwidth,
%        height=.3\textwidth,
%        %title = {Eisengehalt},
%        xlabel = {$I_0$ ($\si{\kilo\gram\meter\squared}$)},
%        %symbolic y coords = {ungewichtet,gewichtet, QtiPlot gewichtet},
%        symbolic y coords = {Referenz Versuchsanleitung,Versuch 3.1.2 Zylinder 2,Versuch 3.1.2 Zylinder,Versuch 3.1.2 Punktmasse,Versuch 3.1.1},
%    ]
%    \addplot+[
%        only marks,error bars/.cd,
%        x dir=both,x explicit,
%        error bar style={line width=0.5pt},
%        ]
%    coordinates {
%        (1.20e-2,Versuch 3.1.1) +- (1.9e-5,0)
%        (1.14e-2,Versuch 3.1.2 Punktmasse) +- (4e-4,0)
%        (1.13e-2,Versuch 3.1.2 Zylinder) +- (4e-4,0)
%        (1.23e-2,Versuch 3.1.2 Zylinder 2) +- (9e-5,0)
%        (1.16e-2,Referenz Versuchsanleitung) +- (0,0)
%    };
%    \end{axis}
%\end{tikzpicture}
%\caption{graphische Darstellung der Ergebnisse zum f\"ur $I_0$}
%\label{fig:resultsI0}
%\end{figure}
