% ---------------------------------------------------------------------------- %
\subsection{Versuch 3.1 -- Spektra der R\"ohren, Analyse von LiF}
\label{subsec:error:spektra}
% ---------------------------------------------------------------------------- %

Zur Rekapitulation nochmals die mittels Regression bestimmten Nullpunktfehler:

\begin{table}[h!]
	\centering
	\small
    \caption{Nullpunktfehler aus Regression}
	\label{tab:nullpunktfehler:LiF}
	\begin{tabular}{lr}
		\toprule

        Anodenmaterial & Nullpunktfehler $\vartheta_0$ \\

        \midrule

        Kupfer &
        $\vartheta_0 = \SI{-0.72925 \pm 0.080931}{\degree}$ \\

        Eisen &
        $\vartheta_0 = \SI{-0.70575 \pm 0.023094}{\degree}$ \\

        Molybd\"an &
        $\vartheta_0 = \SI{-0.70555 \pm 0.068961}{\degree}$ \\

		\bottomrule
	\end{tabular}
\end{table}

Das Ziel ist nun die Bestimmung des Netzebenenabstandes mittels der folgenden Gleichung:

\begin{equation}
	\label{eq:braggLattice}
    d = \frac{n \cdot \lambda}{2 \cdot sin \left( \vartheta_n - \vartheta_0 \right)}
\end{equation}

Wobei  f\"ur  $\vartheta_n$  die   halben  Z\"ahlrohrwinkel  einzusetzen  sind
(abgelesen  aus   den  Messdaten)  und  f\"ur   $\lambda$  die  entsprechenden
Wellenl\"angen  f\"ur  die  charakteristischen Spektrallinien  des  jeweiligen
Anodenmaterials.

Dies liefert f\"ur  jeden Peak jeder Anode einen Wert  f\"ur $d$ (Bestimmt via
Matlab):

\begin{table}[h!]
	\centering
	\small
    \caption{Netzebenenabst\"ande zu den einzelnen Peaks}
	\label{tab:lattices:LiF}
	\begin{tabular}{llr}
		\toprule

        Anodenmaterial & Peak & Netzebenenabstand \\

        \midrule

        Kupfer 								&
		$\beta_1$ 							&
	    \SI{203.9}{\pico\meter}  \\

		&
		$\beta_2$							&
	    \SI{203.5}{\pico\meter}  \\

		&
		$\alpha_1$ &
 	    \SI{204.4}{\pico\meter} \\

		&
		$\alpha_2$ &
 	    \SI{203.3}{\pico\meter} \\

        Eisen &
		$\beta_1$ &
 	    \SI{203.6}{\pico\meter} \\

		&
		$\beta_2$ &
 	    \SI{202.1}{\pico\meter} \\

		&
		$\alpha_1$ &
 	    \SI{204.0}{\pico\meter} \\

		&
		$\alpha_2$ &
 	    \SI{201.9}{\pico\meter} \\

        Molybd\"an &
		$\beta_1$ &
 	    \SI{209.0}{\pico\meter} \\

		&
		$\beta_2$ &
 	    \SI{207.5}{\pico\meter} \\

		&
		$\alpha_1$ &
 	    \SI{205.8}{\pico\meter} \\

		&
		$\alpha_2$ &
 	    \SI{204.6}{\pico\meter} \\

		&
		$\alpha_3$ &
 	    \SI{205.1}{\pico\meter} \\

		\bottomrule
	\end{tabular}
\end{table}

Diese Werte k\"onnen nun gemittelt werden mittels der bekannten Formel:

\begin{equation*}
    \overline{x} = \frac{1}{N} \sum_{i=1}^{N}{x_i}
\end{equation*}

Und ihr Fehler bestimmt werden mittels der Formel:

\begin{equation*}
    s_{\overline{x}} = \sqrt{ \frac{\sum_{1}^{N}{(x_i-\overline{x})^2}}{N \cdot (N-1)}}
\end{equation*}

Da  Matlab   praktische  Befehle  zum  Verk\"urzen   dieser  Prozedur  bereits
integriert hat,  greifen wir darauf zur\"uck:  \code{mean(d)} und \code{std(d)
*  sqrt(1/length(d))},  wobei  \code{d}  der  Resultatvektor  mit  den  Werten
aus  Tabelle  \ref{tab:lattices:LiF}  ist.  \code{std(vector)}  berechnet  die
Standardabweichung der  Werte eines  Vektors, \code{sqrt(1/length(d)}  ist der
Korrekturfaktor, um von der Standardabweichung auf den Fehler des Mittelwertes
der Werte des Vektors zu kommen.

% TODO: Matlab-script

Dies liefert f\"ur den Netzebenenabstand von LiF:

\begin{equation*}
    d_{LiF} = \SI{204.5 \pm 0.6}{\pico\meter}
\end{equation*}


% ---------------------------------------------------------------------------- %
\subsection{Versuch 3.3 -- Planck-Konstante}
\label{subsec:error:planck}
% ---------------------------------------------------------------------------- %

F\"ur diesen Teil  des Versuches werden die Unsicherheiten  aus der Regression
von QtiPlot benutzt.

% ---------------------------------------------------------------------------- %
\subsection{Andere Kristalle}
\label{subsec:error:othercrystals}
% ---------------------------------------------------------------------------- %


Wie   bereits   erw\"ahnt,   wurde   f\"ur  jede   Spektrallinie   von   jedem
Kristall   der   zugeh\"orige   Netzebenenabstand  mittels   der   Bragg'schen
Gleichung  berechnet (in  Matlab). Die  erhaltenen Resultate  sind in  Tabelle
\ref{tab:results:spektra:otherCrystals} zu sehen.

\begin{table}[h!]
    \centering
    \small
    \caption{%
        Die f\"ur die vier Kristalle bestimmten Netzebenenabst\"ande und ihre
		Unsicherheiten.
    }
    \label{tab:results:spektra:otherCrystals}
    \begin{tabular}{lrrrrrrrr}
        \toprule
        &
        \multicolumn{2}{l}{Bergkristall}         &
        \multicolumn{2}{l}{Kalkspat}             &
        \multicolumn{2}{l}{Pyrit}                &
        \multicolumn{2}{l}{Synth. Quartz} \\
        \midrule

        &
        d ($\beta$)  &
        d ($\alpha$) &
        d ($\beta$)  &
        d ($\alpha$) &
        d ($\beta$)  &
        d ($\alpha$) &
        d ($\beta$)  &
        d ($\alpha$) \\

        \midrule


        n = 1              &
        \SI{402.6}{\pico\meter} &
        \SI{406.2}{\pico\meter} &
        \SI{282.7}{\pico\meter} &
        \SI{314.2}{\pico\meter} &
        \SI{331.6}{\pico\meter} &
        \SI{366.5}{\pico\meter} &
        \SI{352.6}{\pico\meter} &
        \SI{391.0}{\pico\meter} \\


        n = 2                   &
        \SI{415.6}{\pico\meter} & % n = 2 bergk
        \SI{415.5}{\pico\meter} & % n = 2 bergk
        \SI{652.8}{\pico\meter} & % n = 2 kalksp
        \SI{684.2}{\pico\meter} & % n = 2 kalksp
                                &
                                &
        \SI{785.4}{\pico\meter} & % n = 2 synth
        \SI{897.1}{\pico\meter} \\ % n = 2 synth

        n = 3                   &
        \SI{408.2}{\pico\meter} & % n = 3 bergk
        \SI{421.5}{\pico\meter} & % n = 3 bergk
                                &
                                &
                                &
                                &
                                &
                                \\

        n = 4                   &
        \SI{406.5}{\pico\meter} & % n = 4 bergk
        \SI{423.6}{\pico\meter} & % n = 4 bergk
                                &
                                &
                                &
                                &
                                &
                                \\

		\midrule

        MW             &
        \SI{408.2}{\pico\meter} &
        \SI{416.7}{\pico\meter} &
        \SI{302.0}{\pico\meter} &
        \SI{310.1}{\pico\meter} &
        \SI{269.8}{\pico\meter} &
        \SI{270.4}{\pico\meter} &
        \SI{251.4}{\pico\meter} &
        \SI{251.1}{\pico\meter} \\

        MW (ges.)      &
        \multicolumn{2}{c}{\SI{412.5}{\pico\meter}} &
        \multicolumn{2}{c}{\SI{306.1}{\pico\meter}} &
        \multicolumn{2}{c}{\SI{251.6}{\pico\meter}} &
        \multicolumn{2}{c}{\SI{270.1}{\pico\meter}} \\

		\midrule

        Fehler         &
         \SI{2.7}{\pico\meter} &
         \SI{3.9}{\pico\meter} &
        \SI{12.8}{\pico\meter} &
         \SI{3.5}{\pico\meter} &
         \SI{0}{\pico\meter} &
         \SI{0}{\pico\meter} &
         \SI{3.0}{\pico\meter} &
         \SI{3.2}{\pico\meter} \\

        Fehler (ges.)    &
        \multicolumn{2}{c}{\SI{2.7}{\pico\meter}} &
        \multicolumn{2}{c}{\SI{5.9}{\pico\meter}} &
        \multicolumn{2}{c}{\SI{0.3}{\pico\meter}} &
        \multicolumn{2}{c}{\SI{1.8}{\pico\meter}} \\

        \bottomrule
    \end{tabular}
\end{table}

\textbf{Anmerkungen:}

\begin{itemize}
	\item
        \textbf{Allgemein:} Die   Resultate  streuen   teilweise  merklich. Es
        dr\"angt  sich  der  Verdacht  auf,   dass  hier  eine  Korrektur  des
        Nullpunktfehlers w\"unschenswert w\"are, wie dies beim LiF-Kristall
		gemacht wurde.
	\item
        \textbf{Bergkristall:} Sieht eigentlich ganz  passabel aus, verglichen
        mit den anderen drei Kristallen.
	\item
        \textbf{Kalkspat:} Hier  sieht  man  einen markenten  Sprung  zwischen
        $n  = 1$  und  $n  = 2$. Ich  f\"uhre  dies  auf die  Kristallstruktur
        zur\"uck,  und  nicht  auf  einen  Mess-  oder  Rechenfehler. Kalkspat
        hat  keine  kubische,  sondern eine  rhombische  Einheitszelle  (siehe
        Anhang Versuchsanleitung). Daher k\"onnen  je nach Einstrahlungswinkel
        verschiedene  Abst\"ande  zwischen  Netzebenen auftreten  (nicht  alle
        Netzebenen unter allen Winkeln sind identisch.
	\item
        \textbf{Pyrit:} Hier  konnten  nur  bei   $n  =  1$  Peaks  detektiert
        werden. Die enorm  kleinen Unsicherheiten  sind daher also  nicht etwa
        das Resultat von hervorragender  Messpr\"azision, sondern das Ergebnis
        einer enorm kleinen Anzahl Messwerte.
	\item
        \textbf{Synthetischer  Quartz:} Auch hier  gelten wieder  die gleichen
        \"Uberlegungen wie beim Kalkspat bez\"uglich verschiedener m\"oglicher
        Netzebenen, nur  dass dieser  Kristall keine rhombische,  sondern eine
        auf Sechsecken aufbauende Gitterstruktur.
\end{itemize}
