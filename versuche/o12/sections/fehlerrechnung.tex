Zur Fehlerrechnung kann man nun  die mittels Regression bestimmten Werte f\"ur
$v_{\mathrm{max}}$  und $k$  mittels der  Formel \ref{eq:volumenstrom:laminar}
von  Seite  \pageref{eq:volumenstrom:laminar}   (laminarer  Fall)  und  Formel
\ref{eq:turbulent:Q}  von  Seite \pageref{eq:turbulent:Q}  (turbulenter  Fall)
\"uberpr\"ufen.

Dazu werden  mittels dieser Formeln die  zu den Resultaten aus  der Regression
geh\"orenden Volumenstr\"ome berechnet und diese anschliessend mit den Angaben
auf dem Durchflussmesser verglichen.

Zur Rekapitulation nochmals die mittels Regression bestimmten Werte:

\begin{align}
    \label{eq:profile:laminar:regression:results:vmax:rekap}
    v_{\mathrm{max, laminar}} &= \SI{1.4026 \pm 0.018346}{\centi\meter\per\second}
    \\
    \label{eq:profile:turb:regression:results:k:rekap}
    k &= 7.8876 \pm 1.5642
    \\
    \label{eq:profile:turb:regression:results:vmax:rekap}
    v_{\mathrm{max, turbulent}} &= \SI{10.793 \pm 0.1572}{\centi\meter\per\second}
\end{align}

% ---------------------------------------------------------------------------- %
\subsection{Laminares Str\"omungsprofil}
\label{subsec:error:laminar}
% ---------------------------------------------------------------------------- %

Wie  in  den  Arbeitsgrundlagen  bei  Abbildung  \ref{fig:einlauf}  auf  Seite
\pageref{fig:einlauf}  erw\"ahnt,  ist  die hier  benutzte  Messleitung  nicht
wirklich  lang   genug,  damit   sich  ein  sauberes,   perfekt  parabolisches
Str\"omungsprofil  im   laminaren  Fall  ausbilden  kann. Aus   der  Abbildung
kann   abgelesen  werden,   dass   f\"ur  den   Fall  einer   \SI{0.9}{\meter}
langen  Messleitung das  Ver\"altnis $\frac{v_{\mathrm{max}}}{v_{\mathrm{m}}}$
ungef\"ahr \num{1.95}  anstatt \num{2} ist. Somit modifizieren  wir die Formel
\ref{eq:volumenstrom:laminar}  leicht, und  setzen  in  den Nenner  \num{1.95}
statt \num{2}, was folgende Formel ergibt:

\begin{equation}
    \label{eq:volumenstrom:laminar:rekap}
    \overline{\dot{V}}_{\mathrm{laminar}} = \frac{\pi \cdot \overline{R}^2 \cdot \overline{v}_{\mathrm{max}}}{1.95} = \SI{0.542}{\liter\per\minute}
\end{equation}

\begin{conditions}
    \overline{R} & \SI{20}{\milli\meter}, Radius der Messleitung \\
    \overline{v}_{\mathrm{max}} & \SI{1.4026}{\centi\meter\per\second} \\
\end{conditions}

Zur   Bestimmung  der   zugeh\"origen  Unsicherheiten   wird  das   Gauss'sche
Fehlerfortpflanzungsgesetz  ben\"otigt. Die  allgemeine   Formel  ist  bereits
in  Gleichung \ref{eq:Gauss}  auf  Seite \pageref{eq:Gauss}  aufgef\"uhrt. Sie
wird  nun angewandt  auf die  Formeln \ref{eq:volumenstrom:laminar:rekap}  und
\ref{eq:turbulent:Q:rekap}.

Dies ergibt f\"ur den Fehler im laminaren Fall:

\begin{equation}
    \label{eq:errorLaminar}
    \begin{split}
        s_{\overline{\dot{V}}} = \sqrt{
            a
            \left(
            \cdot
            \pi
            \cdot
            \overline{R}
            \cdot
            s_{\mathrm{\overline{v}}}
            \right)^2
            +
            b
            \cdot
            \left(
            \pi
            \cdot
            \overline{R}
            \cdot
            s_{\overline{R}}
            \cdot
            v_{\mathrm{max}}
            \right)^2
        }
        =
        \SI{0.0153}{\liter\per\second}
    \end{split}
\end{equation}

\begin{conditions}
    a                & \num{0.26298}                                                         \\
    b                & \num{1.05194}                                                         \\
    \overline{R}     & \SI{20}{\milli\meter}, Radius Messleitung                             \\
    s_{\overline{v}} & \SI{0.018346}{\centi\meter\per\second}, Fehler von $v_{\mathrm{max}}$ \\
    s_{\overline{R}} & \SI{0.25}{\milli\meter} (\emph{Quelle:} Versuchsanleitung)            \\
    v_{\mathrm{max}} & \SI{1.402}{\centi\meter\per\second}                                   \\
\end{conditions}

Die Koeffizienten  $a$ und  $b$ ergeben  sich daraus,  dass im  Nenner \num{2}
durch \num{1.95} ersetzt wurde. Bei \num{2} erg\"abe sich eine ``h\"ubschere''
Formel.

Wie auch  schon in  vorherigen Kapiteln wurde  f\"ur diese  Berechnungen SymPy
benutzt, das  zugeh\"orige Script ist in  Anhang \ref{app:python:errorLaminar}
auf Seite \pageref{app:python:errorLaminar} zu finden.

Zusammengefasst also:

\begin{equation}
    \label{eq:error:laminar:final}
    \dot{V}_{\mathrm{laminar}} = \SI{0.542 \pm 0.0153}{\liter\per\minute}
\end{equation}

% ---------------------------------------------------------------------------- %
\subsection{Turbulentes Str\"omungsprofil}
\label{subsec:error:turbulent}
% ---------------------------------------------------------------------------- %

Der Grundwert f\"ur den Durchfluss errechnet sich gem\"ass:

\begin{equation}
    \label{eq:turbulent:Q:rekap}
    \overline{\dot{V}}_{\mathrm{turbulent}}
        = \frac{
            2
            \cdot
            \pi
            \cdot
            \overline{v}_{\mathrm{max}}
            \cdot
            \overline{R}^2
            \cdot
            \overline{k}^2
        }{
            (\overline{k} + 1)
            \cdot
            (2\overline{k} + 1)
        }
        =
        \SI{6.792}{\liter\per\minute}
\end{equation}

\begin{conditions}
    \overline{v}_{\mathrm{max}} & \SI{10.793}{\centi\meter\per\second}, aus Regression  \\
    \overline{k}                & \num{7.8876}, aus Regression                          \\
    \overline{R}                & \SI{20}{\milli\meter}, aus Versuchsanleitung          \\
\end{conditions}

Die   Anwendung   des   Gauss'schen  Fehlerfortpflanzungsgesetzes   ist   hier
ziemlich   umst\"andlich. Es  muss   nach  der   Geschwindigket,  dem   Radius
und   $k$   abgeleitet   werden. Es   wurde   auch   hier   ein   SymPy-Script
benutzt,  zu  finden  in   Anhang  \ref{app:python:errorTurbulent}  auf  Seite
\pageref{app:python:errorTurbulent}. Der Fehler des Durchflusses ergibt:

\begin{equation}
    \label{eq:errorTurbulent:simple}
    s_{\overline{\dot{V}}} = \sqrt{ A^2 + B^2 + C^2 } = \SI{0.304}{\liter\per\minute}
\end{equation}

\noindent $A$, $B$ und $C$ sind dabei definiert als:

\begin{align}
    \label{eq:errorTurbulent:complicated}
    A
    &=
    \frac{%
        2
        \cdot
        \pi
        \cdot
        \overline{R}^2
        \cdot
        \overline{k}^2
        \cdot
        s_{\overline{v}}
    }{%
        (
            \overline{k}
            +
            1
        )
        (
            2
            \cdot
            \overline{k}
            +
            1
        )
    }
    \\
    B
    &=
    \frac{%
        4
        \cdot
        \pi
        \cdot
        \overline{R}
        \cdot
        \overline{k}^2
        \cdot
        s_{\overline{R}}
        \cdot
        \overline{v}_{\mathrm{max}}
    }{%
        (
            \overline{k}
            +
            1
        )
        (
            2
            \cdot
            \overline{k}
            +
            1
        )
    }
    \\
    C
    &=
    s_{\overline{k}}
    \cdot
    \left(
        \frac{%
            4
            \cdot
            \pi
            \overline{R}^2
            \cdot
            \overline{k}^2
            \cdot
            \overline{v}_{\mathrm{max}}
        }{%
            (
                \overline{k}
                +
                1
            )
            \cdot
            (
                2
                \cdot
                \overline{k}
                +
                1
            )^2
        }
        -
        \frac{%
            2
            \cdot
            \pi
            \cdot
            \overline{R}^2
            \cdot
            \overline{k}^2
            \cdot
            \overline{v}_{\mathrm{max}}
        }{%
            (
                \overline{k}
                +
                1
            )^2
            \cdot
            (
                2
                \cdot
                \overline{k}
                +
                1
            )
        }
        +
        \frac{%
            4
            \cdot
            \pi
            \cdot
            \overline{R}^2
            \cdot
            \overline{k}
            \cdot
            \overline{v}_{\mathrm{max}}
        }{%
            (
                \overline{k}
                +
                1
            )
            \cdot
            (
                2
                \cdot
                \overline{k}
                +
                1
            )
        }
    \right)
\end{align}

\begin{conditions}
    \overline{R}                 & \SI{20}{\milli\meter}                 \\
    \overline{k}                 & \num{7.8876}                          \\
    \overline{v}_{\mathrm{max}}  & \SI{10.793}{\centi\meter\per\second}  \\
    s_{\overline{v}}             & \SI{0.15720}{\centi\meter\per\second} \\
    s_{\overline{R}}             & \SI{0.25}{\milli\meter}               \\
    s_{\overline{k}}             & \num{1.572}                           \\
\end{conditions}

Zusammengefasst also:
\begin{equation}
    \label{eq:error:turbulent:final}
    \dot{V}_{\mathrm{turbulent}} = \SI{6.792 \pm 0.304}{\liter\per\minute}
\end{equation}
